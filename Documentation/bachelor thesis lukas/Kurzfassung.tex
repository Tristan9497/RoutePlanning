\chapter*{Kurzfassung}
\addcontentsline{toc}{chapter}{Kurzfassung}
\label{kurzfassung}

\begin{otherlanguage}{ngerman}

Diese Bachelorarbeit handelt von dem Aufbau, Test und der Qualifizierung eines neuen optischen Messsystems namens \glqq Artificial Eye\grqq (AE). Dieses Projekt wurde von dem Perceived Quality Material Team der Volvo Car Cooperation (VCC) ins Leben gerufen. Sein Ziel ist, die visuellen Aspekte, wie Farbe und Reflexion, auf transparenten, teil-transparenten und lackierten Autoteilen zu studieren. Aus diesem Grund muss das Artificial Eye Messdaten liefern, die die menschliche Wahrnehmung besser darstellen, als es bisherige Messmethoden tun. Nach einer bereits vorangegangenen theoretischen Studie, ist diese Arbeit die erste über den realen Aufbau des Messsystems und wird von darauffolgenden Studenten fortgeführt werden.\\

Um das Erscheinungsbild von Oberflächen zu untersuchen, muss eine effektive Messmethode verwendet werden, um zu verstehen, wie Empfindung und Wahrnehmung mit der Oberfläche in Prozesskontrolle korrelieren. Diese Studie muss in zwei Teile unterteilt werden: \glqq Hard Metrology\grqq und \glqq Soft Metrology\grqq. Der Teil der Hard Metrology wird in dieser Arbeit behandelt und beschreibt das Aufnehmen diskreter Daten wie Oberflächen Reflexion, Streuung und Farbtemperatur. Der Soft Metrology Teil wird aus einer wissenschaftlichen Studie bestehen, die sich mit dem visuellen Aspekt - der Wahrnehmung des Kunden dieser Oberflächen - beschäftigt.\\

Der optische Aufbau besteht aus drei monochromatischen Kameras, einer Farbkamera sowie einem Spektrometer. Mit der Hilfe dieses Aufbaus können dann Eigenschaften wie Oberflächentopologie, Materialdicke (bei durchsichtigen Materialien) und Farbe gemessen werden und mit den Ergebnissen der Soft Metrology Studie korreliert werden. Die Ergebnisse dieser Korrelation werden dann Auskunft über die menschliche visuelle Wahrnehmung der vermessenen Probe geben.
\end{otherlanguage}
