\chapter*{Abstract}
\addcontentsline{toc}{chapter}{Abstract}
\label{abstract}
This bachelor thesis is about the initial setup, testing, qualification and optimisation of a new optical measurement system called "Artificial Eye" (AE). The project was proposed by the Perceived Quality Material Team of Volvo Car Cooperation (VCC). Its aim is to develop visual aspects such as colour and gloss on transparent, translucent and glossy parts used in car bodies. Therefore the Artificial Eye needs to give results that are closer to the human visual perception than common measurement methods can do. With one theoretical study in advance to this thesis, this is the first report about the physical setup of the measurement system and will be continued by following students\cite{Chanal2020}.\\

To study the appearance of surfaces an affective measurement method needs to be used to understand how sensation and perception are correlated with surface appearance and process control. This study is split into two parts: "hard metrology" and "soft metrology". The hard metrology part is covered in this thesis and consists of collecting discrete data of a surface like gloss, haze and colour temperature. The soft metrology part will be a scientific study covering the visual aspect - the customers perception of those surfaces.\\

The optical setup is built around three monochromatic cameras, one colour camera and a spectrometer. With the help of these devices, variables such as surface topology, material thickness (for transparent materials) and colour can be calculated and correlated with the results of the soft metrology study. This correlation will then provide information about the human visual perception of the measured sample.