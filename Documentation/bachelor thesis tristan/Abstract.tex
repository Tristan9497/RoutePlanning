\chapter*{Abstract}
\addcontentsline{toc}{chapter}{Abstract}
\label{abstract}

The vehicle of Aalen University requires a navigation algorithm, to participate in the Carolo-Cup hosted by the Technical University Braunschweig.\\

To extend the work of the existing road detection, this navigation is based on ROS (Robot Operating System) and is using the ROS navigation stack as a core structure. The default configuration of the navigation stack has been modified and newly developed features have been added, in order to achieve lane following, lane preference and obstacle avoidance.
The goals required by the navigation stack are built, by estimating the future course of the road, based on the latest detected road.\\

The given robot has been equipped with a camera, a lidar sensor, an IMU and wheel encoders, and uses differential drive steering.

To test the navigation and its subsystems, a simulated environment has been set up, which allowed fast and efficient changes to the algorithm, robot hardware and environment.\\

The result of these tests suggests, that the navigation stack is a well suited structure for this use case.\\

The established navigation allows to refine the driving behaviors further  in the future, thanks to its flexible approach. Therefore, it is a promising basis for the Carolo-Cup Team.

