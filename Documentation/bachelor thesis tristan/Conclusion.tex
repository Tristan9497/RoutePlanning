\chapter{Conclusion}
\label{Conclusion}
The goal of this thesis was the development of a ROS based navigation for the Carolo-Cup. At the beginning only the road detection existed, which lead to an analysis of the work that has been done in the field of ROS based robot navigation.

The result of this was, that this navigation stack is the best fitting core structure for the ROS based navigation of robots. There have not yet been any publications, regarding the navigation of a Carolo-Cup vehicle using the ROS navigation stack. Therefore it is unsure, if it is usable in the specific conditions of the competition.

Unfortunately, the navigation stack, in its default structure, can not be used for the autonomous navigation on a road. In order to achieve the required driving behaviors, new components have been developed, that allow to preference certain areas over others and extract new goals by estimating the future course of the road.\\

To make quick changes cost and time efficient, the entire robot with its environment has been simulated. Here the main focus was realistic sensor data. This does not only help during changes on the robot or its environment, but simplifies testing as well, since the amount of potential error sources can be limited. Since the simulated sensor data includes common errors, the quality of these signals have been tested.
During testing, it was noticeable, that the raw sensor data did not provide precise enough results, which introduced the necessity of filtering and changes to the hardware of the robot. This improved the data significantly.\\

After implementing the optimizations, the navigation in its entirety has been tested. The robot is now able to follow the road very well, but there are still cases, in which the robot leaves the intended lane based on lack of information.\\

This answers the question, whether the ROS navigation stack is a feasible option for the navigation of a vehicle in the Carolo-Cup. 
As described, the navigation stack is not directly developed for a road like environment, but its flexible approach allowed to modify it in order to satisfies the required navigation behaviors.

The developed navigation is provided in a git repository accessible using the following URL:\\
\url{https://github.com/Tristan9497/RoutePlanning}\\
