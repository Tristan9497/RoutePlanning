\chapter*{Kurzfassung}
\addcontentsline{toc}{chapter}{Kurzfassung}
\label{kurzfassung}

\begin{otherlanguage}{ngerman}
Um an dem Wettbewerb Carolo-Cup, ausgerichtet von der technischen Universität Braunschweig, teilzunehmen, benötigt das Fahrzeug der Hochschule Aalen einen Navigationsalgorithmus.\\

Da die Navigation ebenso wie die bereits vorhandene Spurerkennung auf ROS (Robot Operating System) basieren soll, wird der ROS navigation stack als Grundstruktur verwendet. Um Spurhalten, Spurwahl und Hindernisvermeidung zu gewährleisten, wurde die Standardkonfiguration des navigation stacks im Laufe dieser Arbeit modifiziert und durch weitere Funktionalität ergänzt. Die Ziele für die Navigation werden mit einem geschätzten zukünftigen Spurverlauf ermittelt, der auf der zuletzt erkannten Spur basiert.\\

Der vorgeschriebene Roboter verfügt über eine Kamera, einen Lidar-Sensor, eine IMU und Dreh-Encoder der Räder. Des Weiteren wird ein Differenzialantrieb verwendet.

Eine Simulation des Roboters, mitsamt der Umgebung wurde aufgebaut, um die Navigation zu testen. Dies ermöglichte auch schnelle und effiziente Änderungen des Algorithmus, des Roboters und der Umgebung.\\

Die Testergebnisse zeigen, dass der navigation stack eine geeignete Grundstruktur für die Navigation des Carolo-Cup Fahrzeuges bietet.\\

Der gewählte Aufbau der Navigation ist eine vielversprechende Basis für das Carolo-Cup Team, da zukünftige Ergänzungen zum Fahrverhalten durch die flexible Struktur ermöglicht werden.
\end{otherlanguage}
