\chapter*{Kurzfassung}
\addcontentsline{toc}{chapter}{Kurzfassung}
\label{kurzfassung}

\begin{otherlanguage}{ngerman}

Diese Bachelor-Thesis handelt von der Erstellung eines Konzepts, dem Aufbau und der Entwicklung eines ``Software-Stack'' und dessen Testens für die autonome navigation in einer, durch das Regelwerk des carolo cups beschriebenen, Umgebung.\\

Das Ziel dieses ``Software-Stacks'' ist, der Spur einer Straße zu folgen und dabei potentiellen Hindernissen auf der Straße auszuweichen. Diese Thesis führt die Arbeit von Prof. Hörmann der die von ihm Entwickelte Spurerkennung zur Verfügung stellte und soll in der Zukunft vom Carolo-Cup Team der Hochschule Aalen verwendet werden können. Obwohl der in dieser Arbeit verwendete Roboter nicht konform zum Regelwerk des Carolo-Cups ist, soll der Stack auch für andere Roboter konfigurierbar sein.\\

Der Roboter verfügt über einen Lidar, eine Kamera, Rad-Encoder und einen IMU (inertia measurement unit). Die Daten dieser Sensoren werden gefiltert und dann mit bestehenden ros packages und selbst entwickelten aufbereitet. Die resultierenden Daten werden dann an den Navigation Stack  übergeben, der dann die beste Rute ermittelt.\\

Da zu Beginn dieser Arbeit kein vollständig funktionierender Roboter verfügbar war wurde das Teilthema der Simulation des Roboters mitsamt aller seiner Sensoren und Aktoren in das Thema der Thesis aufgenommen.

\end{otherlanguage}
