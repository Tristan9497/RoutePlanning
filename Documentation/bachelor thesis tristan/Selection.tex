\chapter{Selection}
\label{Selection}

This chapter will cover the package selection and the setup of all nodes in the concept that differ from the recommended setup.


The selection process will mainly focus on the defined requirements.

\section{Simulation}
There are many options, when it comes to robot simulation, which makes a selection mandatory. The chosen Simulator then needs to be configured and equipped with models, sensors and a drive system.

\subsection{Selection}
To begin of the selection process a group of reasonable simulators needs to be collected. The two selected options are Gazebo and V-REP since they are the two most used robotics 3D simulators \cite{SimComp}.\\

Both simulators seem to full fill most of the defined requirements to a certain extend, while Gazebo seems to have a easier installation process and integration into ROS since it is included in the default packages of ROS noetic \cite{ROSPkg} since it developed by the Open Source Robotics Foundation as the default simulator for ROS.

As well as Gazebo V-REP offers plugins and URDF conversion for custom models but an even bigger selection of mobile robot models. Which unfortunately can only be used as examples since the required models are very specific.\\
In contrast to V-REP, Gazebo does not contain an integrated model editor.

Based on computational load compared in the Paper of L. Pitonakova et al\cite{Pitonakova} and the setup differences between both simulators, gazebo will be chosen for this project. But both simulators would have been an excellent choice.

\subsection{Model}
Since Gazebo doesn't have an integrated model editor the freeware blender will be used for the model generation of the environment.

As described in the requirements the robot models will be generated using URDF which will be covered later.

\section{move\_base}

As described in the concept the navigation stack will be used in this thesis. Therefore the action move\_base will be used to get the robot to a given goal.\\
 Move\_base incorporates nav\_core and costmap\_2d and their internal nodes which need to be selected according to the requirements of the navigation.\\ 
The selection of these nodes will be separated into global and local since the individual nodes are highly interconnected and the wanted behavior affects both.

\subsection{global stage}


\subsubsection{planner}

Since the planner needs to adhere to the nav\_core interface \cite{navcore} the following selection of the documentation can be used as a possible selection.

\begin{itemize}
	\item global\_planner
	\item navfn
	\item carrot\_planner
\end{itemize}


Here the choice is fairly easy, since base\_global\_planner is the successor of navfn. It still supports the the behavior of navfn, but it offers more options, such as A* planning algorithm instead of Dijkstra.

\todo{A* and Dijkstra explanation in theoretical}

Carrot planner isn't suited for this use-case since it doesn't fullfil the requirement of being able to cover lane changes since it will not generate plans, that go around obstacles. Instead it will generate a straight path to the goal and shortens the path if the goal is behind or in an obstacle\cite{corrotplanner}.

\subsubsection{costmap}
Since costmap\_2d is embedded in move\_base the package doesn't need to be selected itself. Instead of the package the layers of the individual costmap need to be selected.\\

Judging from the required behavior of the global planner the global costmap needs to have information about lethal obstacles, road markings as well as a defined preference for the right lane.\\

For the road markings and the obstacles a combination of an obstacle layer and an inflation layer will be needed.\\
This results in lethal cells in the costmap for each point of the road detection and the lidar. These points will still have gaps in between each other which will be filled using the inflation layer that inflates every point with a defined cost distribution.\\

Unfortunately there is no costmap plugin that can be used to make certain areas more expensive than others without marking them as lethal. This leads to the following concept of a custom costmap layer.

The only information about where the lane is comes from the road detection, which outputs polynomials that approximate the road markings. The aim is to make a gradually increasing cost from the middle of the right lane to the left road marking. Therefore the left road marking needs to be inflated using dynamically adjustable parameters for the cost distribution.\\
To guarantee that the right lane still has space without cost the right road marking needs to be inflated too. To simplify lane changes in the case of obstacle avoidance these obstacle need to have a cost free zone around them.
The plugin should also feature a reset option as a recovery behaviour

All of this results in a layer with the following abilities

\begin{itemize}
	\item input for points
	\item input of point individual inflation parameters
	\item rasterization of the individual cost distribution
	\item service to reset the cost in the layer
\end{itemize}

\subsection{local stage}
\subsubsection{planner}
In contrast to the global planner there are way more options for the local planner node. Like the global planners the local planners will be selected using the options offered in the following description of the nav\_core.

\begin{itemize}
	\item base\_local\_planner
	\item dwa\_local\_planner
	\item eband\_local\_planner
	\item teb\_local\_planner
	\item mpc\_local\_planner
\end{itemize}\cite{navcore}

To choose the local planner the requirements have to be defined first.\\

Since the global planner is taking care of the obstacles and the roadlanes the local planner has the general task of following the global path and creating a command velocity that is feasible in regards to the kinematics of the robot.\\

Smooth lane changes are highly wanted in this project. This will help the camera and therefore the road detection to keep seeing the road during a lane swap. To achieve this, the planner is supposed to drive close to the global plan, but if it is more efficient smooth out edges of it.\\
Furthermore the local planner needs to have a good performance in tight corridor situations, since those will often be caused by obstacles blocking one lane.\\

During all of this the local planner needs to use the information of lethal obstacles to prevent crashes in its optimized path.\\


Base and DWA are two planners that are included in the navigation stack. Similar to the global planner selection we choose the successor and therefore DWA, which according to Kaiyu Zeng is the general recommendation for mobile robotics platforms. \todo{ros navigation tuning guide kaiyu zheng cite}\\

In addition to DWA we choose one of the planners with an elastic band approach\todo{description elastic band}. The choice between these two is difficult since they both share the same base principle. The decision here is to use teb\_local\_planner for the further selection since it supports carlike robots and therefore satisfies the requirements for the navigation.\\

\subsection{costmap}
The local planner is only supposed to generate velocity commands that lead to no lethal collision. So the local costmap will have the same plugins as the global one without the dynamic inflation layer that is used to guide the robot to the right lane.

\section{Odometry}

\subsection{Encoder}
\subsection{IMU}
\subsection{Improvement using SLAM}

\section{Laser\_Filter}


\section{PoseFinder}
This Node is part of the arlo\_navigation package. Its job is, to find the pose of the next possible goal and send it to the move\_base action client.
Based on the data we gather from the observing sensors the only two sources to gather data about the road situation in front of the robot is the extracted camera data and the data from the last time the robot was in the same area than can be found in the map generated by SLAM.
Since we don't have the map in the first round of the robot gathering goals from the live camera data is more important. 
\subsection{Using current camera data}
The easiest way to get new goals would be to take the last point of the polynomial provided by the road detection as the position and calculate the yaw angle of the new goal with the gradient of the last two.\\

While this is a logical approach in an ideal scenario, it certainly will not work in a realistic one since we can't assure a continuous data stream from the road detection.\\

This is mostly caused by the camera not always seeing enough of the road which can for example be caused by the following reasons:

\begin{itemize}
	\item obstacle covering the markings
	\item while driving a corner the camera isn't always pointed tangential to the curvature
	\item steering during obstacle avoidance
	\item noise in camera data
\end{itemize}


This suggest that a prediction for the possible upcoming road is needed. To a small extend this is provided by the polynomials of the road detection, but the have a restricted domain, after which the error will rapidly increase.

\subsection{Approximations}

Since the road mostly consists of circles of varying radii and origins, it is self evident, that using the polynomials in their restricted domain to represent a section of a circle will give a better estimate.\\
This circle can be calculated using the following linear least-square approach:\\

In theory this approximation should work for almost straight sections as well, but the radius and the origin will trend to infinity and caused by the camera noise and the inaccuracy of the road detection the representation of the road will get worse and worse.\\

This leads to the following linear least square approach for straight lines:\\

Switching between the result of the two approximations will result in the best result for both scenarios. This switching will be triggered by the radii of the approximated circles exceeding a configurable threshold.\\

The next step is a goal extraction from the chosen approximation. This goal will be calculated at a given distance from the robot origin on the approximated route. In contrast to the approximated lines, the circles have a defined angle of trustworthiness. Otherwise small circles would cause the goal to be way of the prediction. The orientation of the goal will then be determined using the differentiation of the function.\\

With the calculated points on either both circles or on both lines the mean can be calculated and represents the new goal for the robot.



\subsection{goal from map}

If the robot is running a SLAM algorithm during the first round the map should be finished once the robot passes its start point. Then the approximation is unnecessary since extracting goals directly from the slam map is more efficient and provides goals that will  always be on the road. Furthermore the distance, at which goals will be found can be increased, which results in higher speeds and/or lower planner frequencies.\\

To find a goal in the SLAM map a circle rasterization algorithm will be used.\\

This algorithm finds every cell on a circular path around the robot and its associated value in the map. The values outside of a given FOV (field of view) can be eliminated. Then the remaining values with a larger likelihood than a configurable threshold will be reduced to one point by taking the mean value of all of them.\\

The orientation of the goal is then determined by using the approximation algorithms.


\section{MarkFreeSpace}

The purpose of this node is to provide data to the SLAM algorithm as well as to the costmaps. This data consists from the points on the polynomials of the road detection in combination with the filtered points of the laser scan.\\

For SLAM and the obstacle layer of the costmap the node simply transforms the data into the same frame and casts it into the form of a sensor\_msgs::PointCloud2.\\

The data of the dynamic cost layer has to contain more information. It is in the form of a sensor\_msgs::PointCloud and contains channel values for point individual inflation radius, min-cost and max-cost. By giving the Layer point individual inflation parameters the node provides information about the cells on the left road marking that need to have inflated cost values and information about the points on the right road marking that should have a clean zone around them. Furthermore the node provides points of obstacles that are on the road and clears the inflated cost around it. Which will allow the global planner to make a plan for passing the obstacle.

\section{SLAM}
There are numerous Lidar based SLAM packages available for ROS, but with the defined restriction of being able to use both, the points extracted from the road detection, as well as the lidar data, most of the lidar based SLAM algorithms wont work since they only accept one input of  the type sensor\_msgs::LaserScan.\\
This rules out the popular options for Lidar based SLAM like gmapping and HectorSLAM and the well documented package Google Cartographer will be used based on its flexibility in regards to robot configurations.\\
This Package comes with an advantage of being based on loop closure which might be usefull when driving rounds in a circuit stile environment. The robot will drive the same route over and over again and thus the map could get more and more reliable over time even though the data is very self simillar and not great for SLAM.\\

Cartographer accepts numerous different input types including both PointCloud2 and LaserScan sensor messages. Additionally Cartographer can use provided odometry, aswell as IMU data to improve the result.\\








