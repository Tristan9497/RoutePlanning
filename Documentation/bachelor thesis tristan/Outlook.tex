\chapter{Outlook}
\label{outlook}

This thesis highlighted existing problems in the established navigation concept, which should be resolved to score high results in the Carolo-Cup. This section will give proposals and ideas for future work on the current state of the navigation.\\

Using the structure of move\_base a custom local and global planner plugin can be developed and exchanged with the current plugins.\\

It could be very interesting to explore different path finding algorithms than the ones offered in the global\_planner plugin. At this point the development of a custom global planner might be able to perform better given the known use case.\\

Seeing the performance of the elastic band in the local planner used in this thesis a potential approach could be to move the task of lane following to the local planner, by deforming the elastic band directly with the polynomials of the road detection and weighting them individually.\\

To decrease the amount of times, where the navigation has to wait until it receives data from the road\_detection, the approximated shape of the road could be drawn into the costmap and overwritten by new incoming data of the road. This could be implemented into the dynamic\_cost\_layer that has been developed during this work.

The exploration of V-REP or its successor CoppeliaSim might be interesting with focus on the sensor plugins, especially with focus on distortion of the camera image.\\

During the research required in this work Nav2 for ROS2, which is currently in development seemed to offer many features, that are not included in the current navigation stack of ROS Noetic. As this thesis is focused on the development based on ROS Noetic this has not been explored yet. Nav2 features a new implementation of a costmap, that features filters usable to define keep-out or slow areas, that might be usable for a cleaner way to guide the robot on the right lane. Therefore further exploration of this might be beneficial.\\

As lidar based SLAM seemed to perform not ideal visual SLAM might offer more precision since the road markings might offer more informations for scan matching than the extracted polynomials. Furthermore functionality to detect, if a map is closed and of a reasonable quality might be usable to end the SLAM procedure, export the current map and using localization only to avoid the problem of the steadily increasing computational burden.